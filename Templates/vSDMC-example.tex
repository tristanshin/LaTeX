\documentclass[mathserif]{beamer}
\usepackage{vSDMC-beamer}
\usepackage{verbatim}

\title[vSDMC Example]{Example vSDMC Beamer}
\subtitle{A walkthrough of some features}
\author{Tristan Shin}
\date{25 Mar 2020}

\begin{document}

\frame{\titlepage}

\section{Basics}

\subsection{Slides}

\begin{frame}[fragile]\frametitle{Slides}
	This is a \vocab{frame} (also known as a \vocab{slide}).
	\pause
	
	You can split a single frame into broken parts for presentation.
	\pause
	
	Just put \verb|\pause| at the point where you want to break the slide.
	\pause
	
	You should not put \verb|\pause| at the end of a slide; see this slide for an example of what happens.
	\pause
\end{frame}

\begin{frame}\frametitle{\texttt{fragile} option}
	Sometimes, using packages (such as \texttt{asymptote} and \texttt{verbatim}) will require the \texttt{fragile} option on any slide using it.
	\pause
	
	For example, the previous slide uses \texttt{verbatim} to display a LaTeX command properly, so it requires the \texttt{fragile} option.
	\pause
	
	This slide does not use any special packages though, so we do not require \texttt{fragile} here.
\end{frame}

\subsection{Formatting}

\begin{frame}\frametitle{\texttt{itemize} and \texttt{enumerate} environments}
	The \texttt{itemize} and \texttt{enumerate} environments are still around.
	\begin{itemize}
		\item Item 1
		\item Item 2
	\end{itemize}
	\begin{enumerate}
		\item Item 1
		\item Item 2
	\end{enumerate}
\end{frame}

\begin{frame}\frametitle{\texttt{itemize} and \texttt{enumerate} environments}
	If you want to break the slide after each item, use the \texttt{<+->} option.
	\pause
	\begin{itemize}[<+->]
		\item Item 1
		\item Item 2
	\end{itemize}
	\begin{enumerate}[<+->]
		\item Item 1
		\item Item 2
	\end{enumerate}
\end{frame}

\begin{frame}[fragile]\frametitle{Emphasis}
	In addition to the usual bold, italics, and underlining, you can do a few additional things:
	\pause
	\begin{itemize}[<+->]
		\item \verb|\alert| gives you \alert{red standout text}.
		\item \verb|\boldalert| gives you \boldalert{the same but in bold}.
		\item \verb|\vocab| gives you \vocab{blue, intended for definitions}.
	\end{itemize}
\end{frame}

\subsection{Drawing}

\begin{frame}[fragile]\frametitle{Drawing}
	If you want to draw, you can use the \texttt{drawing} option in \texttt{vSDMC-beamer}. To do this, replace the line
	
	\verb|\usepackage{vSDMC-beamer}|
	
	with
	
	\verb|\usepackage[drawing]{vSDMC-beamer}|
	
	This will import \texttt{asymptote} as well as some common \texttt{tikz} packages.
	\pause
	
	Alternatively, feel free to import your own packages separately.
\end{frame}

\section{Blocks}

\begin{frame}\frametitle{Blocks}
	Blocks are similar to boxes from the \texttt{tcolorbox} package.
	\begin{block}{A block}
		This is a \vocab{block}. It is quite simple to use.
	\end{block}
	\pause
	\begin{alertblock}{A different block}
		This is an \vocab{alert block}. Its color is different because it is meant to stand out.
	\end{alertblock}
	\pause
	\begin{examples}
		This is an \vocab{example block}. Its color differentiates statements from examples.
	\end{examples}
\end{frame}

\begin{frame}\frametitle{Other blocks}
	There are other blocks modeled after the main three.
	\pause
	\begin{theorem}
		Beamer is cool!
	\end{theorem}
	\begin{corollary}[Wright, 2003]
		Everyone should use beamer!
	\end{corollary}
\end{frame}

\begin{frame}\frametitle{Other blocks}
	Here is a list of environments that act just like \texttt{theorem}:
	\begin{itemize}
		\item \texttt{theorem}, \texttt{corollary}, and \texttt{lemma}
		\item \texttt{proposition}, \texttt{claim}, \texttt{fact}, and \texttt{observation}
		\item \texttt{conjecture} and \texttt{hypothesis}
		\item \texttt{problem}, \texttt{exercise}, and \texttt{question}
		\item \texttt{definition} and \texttt{remark}
	\end{itemize}
	See the next two slides for examples.
\end{frame}

\begin{frame}\frametitle{Other blocks}
	\begin{lemma}[Ruzsa covering lemma]
		Let $X$ and $B$ be subsets of an abelian group. If $|X+B|\leq K|B|$, then there exist $T\subseteq X$ with $|T|\leq K$ such that $X\subseteq T+B-B$.
	\end{lemma}
	\begin{claim}
		There are finitely many countries.
	\end{claim}
	\begin{conjecture}[Riemann hypothesis]
		For all $n\in\NN$,
		\[
			\sigma(n)\leq H_n+(\ln H_n)e^{H_n}.
		\]
	\end{conjecture}
	Note that \texttt{conjecture} and \texttt{hypothesis} are a lighter shade of blue.
\end{frame}

\begin{frame}\frametitle{Other blocks}
	\begin{problem}
		Determine all possible values of the expression
		\[
			A^3+B^3+C^3-3ABC
		\]
		where $A$, $B$, and $C$ are nonnegative integers.
	\end{problem}
	\begin{definition}[Continuity]
		Let $(X,d_X)$ and $(Y,d_Y)$ be metric spaces and $f\colon X\to Y$ be a function. For $x_0\in X$, we say that \vocab{$f$ is continuous at $x_0$} if for all $\eps>0$, there exists a $\delta>0$ such that $d_X(x,x_0)<\delta$ implies $d_Y(f(x),f(x_0))<\eps$.
	\end{definition}
	Note that \texttt{definition} and \texttt{remark} are black.
\end{frame}

\begin{frame}\frametitle{Proof block}
	The \texttt{proof} environment has also been transformed into a block.
	\begin{proof}
		Left as an exercise to the reader.
	\end{proof}
	\pause
	This environment also has an option to let you retitle the proof.
	\begin{proof}[Proof sketch]
		Still left as an exercise to the reader.
	\end{proof}
\end{frame}

\begin{frame}\frametitle{Proof block}
	A common mistake in spacing is to leave the $\square$ dangling.
	\begin{proof}
		We have that
		\begin{align*}
			\sec^2\theta-1 &= \frac{1-\cos^2\theta}{\cos^2\theta} \\
			&= \frac{\sin^2\theta}{\cos^2\theta} \\
			&= \tan^2\theta
		\end{align*}
	\end{proof}
\end{frame}

\begin{frame}[fragile]\frametitle{Proof block}
	To fix this, use \verb|\qedhere| where the end of your last line is.
	\begin{proof}
		We have that
		\begin{align*}
			\sec^2\theta-1 &= \frac{1-\cos^2\theta}{\cos^2\theta} \\
			&= \frac{\sin^2\theta}{\cos^2\theta} \\
			&= \tan^2\theta
			\qedhere
		\end{align*}
	\end{proof}
	\pause
	You will need this when you use \texttt{itemize}, \texttt{enumerate}, or any form of display math environment to end your proof.
\end{frame}

\section{Preamble}

\begin{frame}\frametitle{Shortcuts}
	There are several shortcuts that are included in \texttt{vSDMC-beamer}. Check them out in the file!
	\pause
	
	Feel free to import other packages in the preamble, or make your own shortcuts. Common packages to import include
	\begin{itemize}
		\item \texttt{asymptote}
		\item \texttt{tikz-cd} (highly recommend checking out \texttt{tikzducks} too for fun)
	\end{itemize}
	\pause
	Keep in mind that the more packages you import, the slower your presentation will compile. This is especially important for live-TeXing.
	\pause
	
	For example, this presentation takes about 3 seconds to compile (only additional import is \texttt{verbatim}). This is probably fine for live-TeXing.
\end{frame}

\section{Conclusion}

\begin{frame}\frametitle{Good luck!}
	One final note: LaTeX table of contents requires two compilations to update correctly. So if you create a new frame or change references, the first compilation after this may have incorrect referencing or page numbers. Be wary of this.
	\pause
	
	Have fun creating your own beamer! For questions, contact me at \email{shint@mit.edu}.
\end{frame}

\end{document}

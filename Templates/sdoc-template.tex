%%%%%%%%%%%%%%%%%%%%%%%%%%%%%%%%%%%%%%%%%%%%%%
%%                                          %%
%% USE THIS FILE TO WRITE YOUR SOLUTIONS    %%
%%                                          %%
%% You can copy and paste this              %%
%% template file into Overleaf or           %%
%% any other LaTeX editor.                  %%
%%                                          %%
%% You should NOT submit this file          %%
%% with your solutions. You only need to    %%
%% submit the output PDF file.              %%
%%                                          %%
%% Make sure you compile this file TWICE    %%
%% to make sure the page numbers            %%
%% are correct!                             %%
%%                                          %%
%% Make sure fancyhdr.sty is installed.     %%
%%                                          %%
%% If the compiler hangs or asks for        %%
%% fancyhdr.sty when you compile, then      %%
%% email us for LaTeX support.              %%
%%                                          %%
%% If you have any questions or problems    %%
%% using this file, or with LaTeX in        %%
%% general, feel free to email us at        %%
%% sandiegoonlinecontest@gmail.com          %%
%%                                          %%
%% Author: Kevin Ren                        %%
%% Edited by: Tristan Shin                  %%
%%                                          %%
%%%%%%%%%%%%%%%%%%%%%%%%%%%%%%%%%%%%%%%%%%%%%%

%%%%%%%%%%%%%%%%%%%%%%%%%%%%%%%%%%%%%%%%%%%%%%
%%%%%%%%%%%%%%%%%%%%%%%%%%%%%%%%%%%%%%%%%%%%%%
%%%%%%%%%%%%%%%%%%%%%%%%%%%%%%%%%%%%%%%%%%%%%%
%% DO NOT ALTER THE FOLLOWING LINES
\documentclass[letter, 12pt]{article}
\usepackage{amsmath, amsthm, amssymb, amsfonts}
\usepackage{graphicx}
\newcommand{\name}[1]
{\newcommand{\studentname}{#1}}
\usepackage[margin = 1in]{geometry}
\usepackage{fancyhdr}
\pagestyle{fancy}
\setlength{\headheight}{42pt}
\lhead{San Diego Online Contest}
\rhead{\studentname{} \\
Problem \theqnumber{} \\
Page \thepage{} of \number\value{numpages}}
\lfoot{}\cfoot{}\rfoot{}
\newlength\tindent
\setlength{\tindent}{\parindent}
\setlength{\parindent}{0pt}
\renewcommand{\indent}{\hspace*{\tindent}}
\setlength{\parskip}{1em}
\usepackage{refcount}
\newcounter{qnumber}
\newcounter{numpages}
\newenvironment{solution}[1]
{\setcounter{qnumber}{#1}
\setcounter{page}{1}
\setcounter{numpages}
{\getpagerefnumber{\theqnumber}}}
{\label{\theqnumber}\eject}
\DeclareMathOperator{\lcm}{lcm}
%% DO NOT ALTER THE ABOVE LINES
%%%%%%%%%%%%%%%%%%%%%%%%%%%%%%%%%%%%%%%%%%%%%%
%%%%%%%%%%%%%%%%%%%%%%%%%%%%%%%%%%%%%%%%%%%%%%
%%%%%%%%%%%%%%%%%%%%%%%%%%%%%%%%%%%%%%%%%%%%%%

%% Enter your name here
%% Example: \name{Jane Doe}
\name{Your Name Here}

%% DO NOT ALTER THE FOLLOWING LINE
\begin{document}
%% DO NOT ALTER THE PREVIOUS LINE

%%%%%%%%%%%%%%%%%%%%%%%%%%%%%%%%%%%%%%%%%%%%%%
%%                                          %%
%% All solutions go in between the          %%
%% \begin{solution} and the \end{solution}  %%
%% corresponding to the problem number.     %%
%%                                          %%
%% For example, suppose Problem 2 is        %%
%% "Solve for x: x + 3 = 5."                %%
%% Your solution would look like this:      %%
%%                                          %%
%% \begin{solution}{2}                      %%
%% If $x+3=5$, then subtract 3 from both    %%
%% sides of the equation to get $x=5-3=2$,  %%
%% so the solution is $x=2$.                %%
%% \end{solution}                           %%
%%                                          %%
%% If you are not writing a solution for    %%
%% a problem, leave that solution field     %%
%% UNTOUCHED or BLANK!                      %%
%%                                          %%
%%%%%%%%%%%%%%%%%%%%%%%%%%%%%%%%%%%%%%%%%%%%%%

\begin{solution}{1} 
	%% Solution to Problem 1 goes here
	This is a sample solution
	that you should replace.
\end{solution}

\begin{solution}{2}
	%% Solution to Problem 2 goes here
\end{solution}

\begin{solution}{3}
	%% Solution to Problem 3 goes here
\end{solution}

%% DO NOT ALTER THE FOLLOWING COMMAND
\end{document}
%% DO NOT ALTER THE PREVIOUS COMMAND
